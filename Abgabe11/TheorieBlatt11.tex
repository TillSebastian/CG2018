% header
\documentclass[10pt,a4paper]{article}

\usepackage[utf8]{inputenc}
\usepackage{hyperref}
\usepackage{amssymb}
\usepackage{xcolor}
\usepackage{ngerman}
\usepackage{mathtools}
\selectlanguage{german} 
\usepackage[german,onelanguage]{algorithm2e} %for psuedo code
% the document
\begin{document}

% Ersetzt in den eckigen Klammern bitte die Übungsnummer.
\title{Abgabe - Übungsblatt [$11$]\\
\small{Einführung in die Computergraphik und Visualisierung}}
\author{ [Till Sebastian] \and [Felix Grefe] \and [Marius Rometsch]}
\date{\today}
\maketitle



\section*{Aufgabe 1}
\subsection*{a)}
 \[ f(x) =
\begin{cases}
xe^{-x}       & \quad \text{, } x\geq 0\\
0   & \quad \text{,sonst } 
\end{cases}
\]
$
\hat{f}(\omega)= \int_{-\infty}^{\infty} f(t) e^{-2\pi i \omega t}\;  \mathrm{d} t
\\=\int_{0}^{\infty} te^{-t} e^{-2\pi i \omega t}\;  \mathrm{d} t
\\=[\frac{e^{(-1-2i\pi\omega)t}(2i\pi\omega t+t+1)}{(-2\pi\omega+i)^2}]_0^\infty
\\=\frac{1}{(-2\pi\omega+i)^2}(e^{-\infty-i\infty}(i\infty+\infty+1)-(e^0*1))
\\=\frac{1}{(-2\pi\omega+i)^2}(0-1)
\\=\frac{-1}{i^2-4\pi\omega i+4\pi^2\omega^2}
\\=\frac{-1}{-1-4\pi\omega i+4\pi^2\omega^2}
\\=\frac{1}{1+4\pi\omega i-4\pi^2\omega^2}
\\=\frac{1}{(1+2\pi\omega i)^2}
$




\subsection*{b)}
$
f(x)=e^{-a|x|}
\\\hat{f}(\omega)= \int_{-\infty}^{\infty} f(t) e^{-2\pi i \omega t}\;  \mathrm{d} t
\\=2\int_{0}^{\infty} e^{-at} e^{-2\pi i \omega t}\;  \mathrm{d} t
\\=2\int_{0}^{\infty} e^{-at-2\pi i \omega t}\;  \mathrm{d} t
\\=\frac{2}{a+2i\pi\omega}[-e^{t(-(a+2i\pi\omega))}]_0^\infty
\\=\frac{2}{a+2i\pi\omega}[-e^{-\infty-i\infty}+e^0]
\\=\frac{2}{a+2i\pi\omega}
$
\subsection*{c)}

\end{document}
