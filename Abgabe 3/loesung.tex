% header
\documentclass[10pt,a4paper]{article}

\usepackage[utf8]{inputenc}
\usepackage{hyperref}
\usepackage{amssymb}
\usepackage{ngerman}
\usepackage{mathtools}
% the document
\begin{document}

% Ersetzt in den eckigen Klammern bitte die Übungsnummer.
\title{Abgabe - Übungsblatt [$3$]\\
\small{Einführung in die Computergraphik und Visualisierung}}
\author{ [Till Sebastian] \and [Felix Grefe] \and [Marius Rometsch]}
\date{\today}
\maketitle

\section*{Erste Übung}
$
p_1 =\begin{pmatrix}
	7 \\
	 1,5 \\
	 2 
\end{pmatrix}$
$
p_2 =\begin{pmatrix}
4 \\
0 \\
1 
\end{pmatrix}
$

Alle Vektoren mit Null an der vierten Stelle können nicht auf den $R^3$ abgebildet werden und definieren die uneigentliche Hyperebene.
\section*{Zweite Übung}
Nach der Vorlesung:\newline
$ p'=(vl^t-l^tvI)*p ; v=(0,-y_0,1)$
\newline Bildebene: x-Achse $g:y=0$\newline
$\Longrightarrow l= [0 1 0]^t$\newline
$\Longrightarrow vl^t=\begin{bmatrix}
	0 & 0 & 0\\
	 0 & -y_0 & 0 \\
	 0 & 1 & 0 
\end{bmatrix}$
\newline und $ l^t vI=(\begin{pmatrix}
   0 & 1 & 0 \end{pmatrix}*\begin{pmatrix}
   0 & -y_0 & 1 \end{pmatrix} =-y_0*I$\newline
   $\Longrightarrow (vl^t-l^tvI)*p= \begin{bmatrix}
   0 & 0 & 0 \\
   0 & -y_0 & 0 \\
   0 & 1 & 0 \end{bmatrix}-y_0\begin{bmatrix}
   1 & 0 & 0 \\
   0 & 1 & 0 \\
   0 & 0 & 1
   \end{bmatrix})*p
   $ \newline$=\underbrace{\begin{bmatrix}
   -y_0 & 0 & 0 \\
   0 & -2y_0 & 0 \\
   0 & 1 & -y_0 \end{bmatrix}}*\begin{pmatrix}
   x\\
   y\\
   1\end{pmatrix}$\newline
   Perspektivische Transformation


\end{document}
