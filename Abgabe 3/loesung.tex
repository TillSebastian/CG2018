% header
\documentclass[10pt,a4paper]{article}

\usepackage[utf8]{inputenc}
\usepackage{hyperref}
\usepackage{amssymb}
\usepackage{ngerman}
\usepackage{mathtools}
% the document
\begin{document}

% Ersetzt in den eckigen Klammern bitte die Übungsnummer.
\title{Abgabe - Übungsblatt [$3$]\\
\small{Einführung in die Computergraphik und Visualisierung}}
\author{ [Till Sebastian] \and [Felix Grefe] \and [Marius Rometsch]}
\date{\today}
\maketitle

\section*{Erste Übung}
$
p_1 =\begin{pmatrix}
	7 \\
	 1,5 \\
	 2 
\end{pmatrix}$
$
p_2 =\begin{pmatrix}
4 \\
0 \\
1 
\end{pmatrix}
$

Alle Vektoren mit Null an der vierten Stelle können nicht auf den $R^3$ abgebildet werden und definieren die uneigentliche Hyperebene.
\section*{Zweite Übung}
Noch eine Lösung \ldots

\end{document}
