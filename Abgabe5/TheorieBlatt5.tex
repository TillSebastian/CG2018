% header
\documentclass[10pt,a4paper]{article}

\usepackage[utf8]{inputenc}
\usepackage{hyperref}
\usepackage{amssymb}
\usepackage{ngerman}
\usepackage{mathtools}
\selectlanguage{german} 
\usepackage[german,onelanguage]{algorithm2e} %for psuedo code
% the document
\begin{document}

% Ersetzt in den eckigen Klammern bitte die Übungsnummer.
\title{Abgabe - Übungsblatt [$5$]\\
\small{Einführung in die Computergraphik und Visualisierung}}
\author{ [Till Sebastian] \and [Felix Grefe] \and [Marius Rometsch]}
\date{\today}
\maketitle

\section{Perspektivische Projektion}
\subsection{Variablen von P}
\begin{itemize}
\item $n=$near ; Abstand zur nahen Projektionsebene
\item $f=$far ; Abstand zur fernen Projektionsebene
\item $r=$right ; Rechter Clippingrand der nahen Projetionsebene
\item $l=$left ; Linker Clippingrand der nahen Projetionsebene
\item $t=$top ; Oberer Clippingrand der nahen Projetionsebene
\item $b=$bottom ; Unterer Clippingrand der nahen Projetionsebene
\end{itemize}


\subsection{Wieso $w=-1$}

\subsection{Asymmetrisches Frustum}


\end{document}
