% header
\documentclass[10pt,a4paper]{article}

\usepackage[utf8]{inputenc}
\usepackage{hyperref}
\usepackage{amssymb}
\usepackage{ngerman}
\usepackage{mathtools}
\usepackage{amsfonts}
% the document
\begin{document}

% Ersetzt in den eckigen Klammern bitte die Übungsnummer.
\title{Abgabe - Übungsblatt [$6$]\\
\small{Algo2}}
\author{ [Till Sebastian]}

\maketitle

\section*{6.1}
Wäre L\textsubscript{3} durch M\textsubscript{$3$} entscheidbar, dann gäbe es eine M\textsubscript{H} die H entscheidet von folgender Form:
M\textsubscript{H} durchgeht folgende Schritte:
\begin{enumerate}
	\item Prüfe ob die Eingabe a von der Form $\langle M\rangle $$\omega$ ist, wenn nicht lehne ab
	\item Berechne Die Gödelnummer einer DTM die folgendes tut:
	\begin{enumerate}
		\item speichere die Eingabe b
		\item simuliere M auf $\omega$
		\item akzeptiere falls die Eingabe bx 0,1,oder 01 ist
	\end{enumerate}
	
	\item Simuliere nun M\textsubscript{3} auf diese Maschine
\end{enumerate}
Wenn nun also M auf $\omega$ hält akzeptiert M\textsubscript{H} und wenn M nicht hält akzeptiert M\textsubscript{H} diese Eingabe nicht da M\textsubscript{3} nicht akzeptiert.
Also entscheidet M\textsubscript{H} H und das geht wie wir wissen nicht.
\\L\textsubscript{3} ist also unentscheidbar.



\section*{6.2}


\section*{6.3}
$T:=\{\langle M\rangle|M \text{ ist DTM, die auf jeder Eingabe hält}\}$
\\Wir wissen aus dem Skript, dass diese Sprache nicht rekursiv ist.
\\Sei
 $
 f(\omega) =
\begin{cases}
\omega      & \quad \text{falls } \omega \text{ nicht von der Form } \langle M\rangle\text{ ist }\\
\langle \bar{M} \rangle  & \quad \text{wenn } \omega \text{ von der Form } \langle M\rangle\text{ ist }
\end{cases}
$
\\$\langle \bar{M} \rangle$ arbeitet wie folgt:
\begin{enumerate}
	\item Simuliere M mit Eingabe x
	\item falls M x akzeptiert , akzeptiere x
	\item falls x nicht prim, lehne x ab
\end{enumerate}
Falls nun $\omega$ nicht von der Form $\langle M\rangle$ ist so gilt auch $\omega \notin T$ und $ \omega = f(\omega) \notin A_\mathbb{P}.$
\\Es gilt aber:
\\$\omega =\langle M\rangle \in T \Leftrightarrow f(\omega)=f(\langle M\rangle)=\langle \bar{M} \rangle \in A_\mathbb{P}
$ 
\\Denn aus $\omega =\langle M\rangle \in T$ folgt, dass $ M $ hält und nach Definition folgt dass $ \bar{M} $ $\mathbb{P}$ entscheidet.
\\ Andersherum wissen wir ,dass aus$\langle \bar{M} \rangle \in A_\mathbb{P}$ folgt, dass  $ \bar{M} $ $\mathbb{P}$ entscheidet, dann muss M aber halten und damit ist $\omega =\langle M\rangle \in T$.
\\T kann also auf$A_\mathbb{P}$ reduziert werden.$A_\mathbb{P}$ ist also nicht rekursiv.


\end{document}