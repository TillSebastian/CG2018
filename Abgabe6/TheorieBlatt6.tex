% header
\documentclass[10pt,a4paper]{article}

\usepackage[utf8]{inputenc}
\usepackage{hyperref}
\usepackage{amssymb}
\usepackage{ngerman}
\usepackage{mathtools}
\selectlanguage{german} 
\usepackage[german,onelanguage]{algorithm2e} %for psuedo code
% the document
\begin{document}

% Ersetzt in den eckigen Klammern bitte die Übungsnummer.
\title{Abgabe - Übungsblatt [$6$]\\
\small{Einführung in die Computergraphik und Visualisierung}}
\author{ [Till Sebastian] \and [Felix Grefe] \and [Marius Rometsch]}
\date{\today}
\maketitle



\section{Texturen}
\subsection{Bayrenzentrische Koordinaten}

$
A =\begin{pmatrix}
1 \\
5 \\

\end{pmatrix}
$$
B =\begin{pmatrix}
1 \\
1 \\

\end{pmatrix}
$$
C =\begin{pmatrix}
	3 \\
	1\\
	 
\end{pmatrix}
$\\
Es muss gelten:

\begin{itemize}
 \item  x = $\alpha$ * A + $\beta$ * B + $\gamma$ * C
 \item  1 = $\alpha$ + $\beta$ + $\gamma$
\end{itemize}
Aus diesen Bedingungen ergibt sich ein LGS mit den baryzentrischen Koordinaten von x als Loesungen:
\\
$\implies$
$
x =\begin{pmatrix}
0,5 \\
0,25 \\
0,25 
\end{pmatrix}
$
Es gilt $x_1*c(A)+x_2*c(B)+x_3*c(C) = UV$$
\\f_1 =\begin{pmatrix}
0 \\
0,25 
\end{pmatrix}
$ $
f_2 =\begin{pmatrix}
0,05 \\
0,2 
\end{pmatrix}
$ 
$
f_3 =\begin{pmatrix}
0,15  \\
0,175
\end{pmatrix}
$
\\
$
UV =\begin{pmatrix}
0,2 \\
0,625
\end{pmatrix}
$ 
\subsection{nearest neighbor}

C\textsubscript{tex}(0.2,0.625)=C{(0,2*1+1/2)mod 1,{(0,625+0,5) mod 1}}=C(0,1)
\\Der Punkt ist grau
\subsection{Bilineare Interpolation}
$\hat{u}=0,2-0    
\\\hat{v}=0,625-0
\\\implies
\\C_j(u,v)=0,2*C[0+1,0+j]+0,8*C[0,0+j]
\\\implies
\\C_0(u,v)=0,2*C[1,0]+0,8*C[0,0]=C \begin{pmatrix}
0,2 \\
0
\end{pmatrix}
\\C_1(u,v)=0,2*C[1,1]+0,8*C[0,1]=C\begin{pmatrix}
0,2 \\
1
\end{pmatrix}
\\\implies
\\C\textsubscript{tex}(u,v)=0,625*C_1(u,v)+0,375*C_0(u,v)
\\\implies
\\C\textsubscript{tex}(u,v)=0,625*C\begin{pmatrix}
0,2 \\
1
\end{pmatrix}+0,375*C\begin{pmatrix}
0,2 \\
0
\end{pmatrix}
\\\implies
\\C\textsubscript{tex}(u,v)=0,625*C\begin{pmatrix}
0,2 \\
0,625
\end{pmatrix}
$
\\Die Farbe ist also auch grau
\end{document}
