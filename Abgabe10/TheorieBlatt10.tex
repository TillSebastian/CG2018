% header
\documentclass[10pt,a4paper]{article}

\usepackage[utf8]{inputenc}
\usepackage{hyperref}
\usepackage{amssymb}
\usepackage{xcolor}
\usepackage{ngerman}
\usepackage{mathtools}
\selectlanguage{german} 
\usepackage[german,onelanguage]{algorithm2e} %for psuedo code
% the document
\begin{document}

% Ersetzt in den eckigen Klammern bitte die Übungsnummer.
\title{Abgabe - Übungsblatt [$10$]\\
\small{Einführung in die Computergraphik und Visualisierung}}
\author{ [Till Sebastian] \and [Felix Grefe] \and [Marius Rometsch]}
\date{\today}
\maketitle



\section*{Aufgabe 1}
\subsection*{a)}
$A(x)=U(x)+t
\\A(\sum_{i=0}^{n}b_i B^n _i(t))
\\=U(\sum_{i=0}^{n}b_i B^n _i(t))+t
\\=\sum_{i=0}^{n}(U (b_i B^n _i(t))+B^n _i(t)t)$ Dies folgt aus der Partition von Eins
$
\\=\sum_{i=0}^{n}(B^n _i(t)(U (b_i)+B^n _i(t)t))
\\=\sum_{i=0}^{n}(A(b_i) B^n _i(t))
$


\subsection*{b)}



\end{document}
