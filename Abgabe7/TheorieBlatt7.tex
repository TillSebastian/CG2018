% header
\documentclass[10pt,a4paper]{article}

\usepackage[utf8]{inputenc}
\usepackage{hyperref}
\usepackage{amssymb}
\usepackage{ngerman}
\usepackage{mathtools}
\selectlanguage{german} 
\usepackage[german,onelanguage]{algorithm2e} %for psuedo code
% the document
\begin{document}

% Ersetzt in den eckigen Klammern bitte die Übungsnummer.
\title{Abgabe - Übungsblatt [$7$]\\
\small{Einführung in die Computergraphik und Visualisierung}}
\author{ [Till Sebastian] \and [Felix Grefe] \and [Marius Rometsch]}
\date{\today}
\maketitle



\section*{Reflexionsmodelle}
\subsection*{a)}
Ein BRDF ist eine Funktion die für Oberflächen von Materialien das Reflexionsverhalten berechnet.
BRDF sind Wahrscheinlichkeitsverteilungen, dass bedeutet sie gibt für einkommende Strahlen an wie wahrscheinlich sie in eine bestimmte Richtung reflektiert werden.
Man kann bei einem BRDF einfach die Einfall- und Ausfallsrichtung der Strahlen vertauschen, dies ist MÖglich durch die Helmholz-Reziprozität der Gleichung.
\subsection*{b)}
Mit BSSRDF kann man auch berechnen wie sich Strahlen verhalten die in das Material eindringen und an einem anderen Punkt wieder austreten.
\subsection*{c)}
Das Blinn-Phong-Modell basiert auf dem Phong-Modell ist aber schneller zu berechnen, da man nicht für jede Normale die Reflexionsrichtung neu berechnen muss.
\subsection*{d)}
Das Cosine-Lobe-BRDF ist Energie erhaltend im Gegensatz zum Blinn-Phong-Modell.
\subsection*{e)}

\end{document}
