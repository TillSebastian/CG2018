% header
\documentclass[10pt,a4paper]{article}

\usepackage[utf8]{inputenc}
\usepackage{hyperref}
\usepackage{amssymb}
\usepackage{ngerman}
\usepackage{mathtools}
\selectlanguage{german} 
\usepackage[german,onelanguage]{algorithm2e} %for psuedo code
% the document
\begin{document}

% Ersetzt in den eckigen Klammern bitte die Übungsnummer.
\title{Abgabe - Übungsblatt [$4$]\\
\small{Einführung in die Computergraphik und Visualisierung}}
\author{ [Till Sebastian] \and [Felix Grefe] \and [Marius Rometsch]}
\date{\today}
\maketitle

\section{Clipping-Algorithmen}
\subsection{n-dimensionaler Liang-Barsky-Algorithmus}
\begin{itemize}
\item Im n-dimensionalen Raum werden Halbräume durch (n-1)-dimensionale Objekte definiert
\end{itemize}


\subsection{Sutherland-Hodgman-Algorithmus}
\begin{algorithm}[H]
 \KwData{Liste p $[P_1:P_n]$, Viewport v}
 \KwResult{In Viewport geclipptes Polygon aus der Punktliste pErg  }
 \caption{Sutherland-Hodgman-Algorithmus}
 Liste pErg\;
 \For{Jeden Eckpunkt E des Viewports v: index i}{
 	\For{Jeden Polygon-Eckpnkt P: index j}{
 		
 	   \uIf{$\overrightarrow{P_j P_{(j+1)mod n}}$ auf der sichtbaren Seite von $\overrightarrow{E_i E_{(i+1)mod4}}$}{
			$P_{(j+1)mod n}$ zu pErg hinzufügen\;
		}
		\uElseIf{$\overrightarrow{P_j}$ auf der sichtbaren Seite von $\overrightarrow{E_i E_{(i+1)mod4}}$ und $\overrightarrow{P_{(j+1)mod n}}$ nicht}{
			Schnittpunkt I von $\overrightarrow{P_j P_{(j+1)mod n}}$ mit $\overrightarrow{E_i E_{(i+1)mod4}}$ zu pErg hinzufügen\;
		}
		\uElseIf{$\overrightarrow{P_j P_{(j+1)mod n}}$ auf der nicht sichtbaren Seite von $\overrightarrow{E_i E_{(i+1)mod4}}$}{
			Schnittpunkt I von $\overrightarrow{P_j P_{(j+1)mod n}}$ mit $\overrightarrow{E_i E_{(i+1)mod4}}$ zu pErg hinzufügen\;
		}
		\Else{Schnittpunkt I von $\overrightarrow{P_j P_{(j+1)mod n}}$ mit $\overrightarrow{E_i E_{(i+1)mod4}}$ und $\overrightarrow{P_{(j+1)mod n}}$ zu pErg hinzufügen\;}
	} 
	Ersetze p durch pErg\;
 }
 Gebe pErg zurück\;
\end{algorithm}


\section{Baryzentrische Koordinaten und Farbinterpolation}

\subsection{Baryzentrische Koordinaten}

$
v_1 =\begin{pmatrix}
1 \\
0 \\
0 
\end{pmatrix}
$$
v_2 =\begin{pmatrix}
0 \\
1 \\
0 
\end{pmatrix}
$$
v_3 =\begin{pmatrix}
	0 \\
	0 \\
	1 
\end{pmatrix}
$\\
$
Q =\begin{pmatrix}
2/3 \\
1/3 \\
0 
\end{pmatrix}
$
ist der Schnittpunkt der Geraden von $v_3$ nach x und $v_1$ nach $v_2$.
\\Dann gilt noch $0,1^2+0,6^2$=0,26 (Die Strecke von$v_3$ nach x)
\\und $0,4^2+(2/30)^2$=0,16444444...(Die Strecke von x nach Q)
\begin{itemize}
 \item  0,26+0,1644...=0,42444...
 \item  0,42444.../0,16444...=0,3874346
\end{itemize}
$\implies$
$
x =\begin{pmatrix}
0,40837697 \\
0,2041885 \\
0,3874346 
\end{pmatrix}
$



\subsection{Farbinterpolation}

Es gilt $x_1*c(v_1)*x_2*c(v_2)+x_3*c(v_3)$$
\\f_1 =\begin{pmatrix}
0 \\
0,2041885... \\
0
\end{pmatrix}
$ $
f_2 =\begin{pmatrix}
0 \\
0,2041885 \\
 0,1020942
\end{pmatrix}
$ 
$
f_3 =\begin{pmatrix}
0,3874346  \\
0,19371728 \\
0
\end{pmatrix}
$
\\
$
farbe =\begin{pmatrix}
0,3874346  \\
0,79581152\\
 0,1020942
\end{pmatrix}
$ 

\end{document}
