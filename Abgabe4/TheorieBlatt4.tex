% header
\documentclass[10pt,a4paper]{article}

\usepackage[utf8]{inputenc}
\usepackage{hyperref}
\usepackage{amssymb}
\usepackage{ngerman}
\usepackage{mathtools}
\selectlanguage{german} 
\usepackage[german,onelanguage]{algorithm2e} %for psuedo code
% the document
\begin{document}

% Ersetzt in den eckigen Klammern bitte die Übungsnummer.
\title{Abgabe - Übungsblatt [$4$]\\
\small{Einführung in die Computergraphik und Visualisierung}}
\author{ [Till Sebastian] \and [Felix Grefe] \and [Marius Rometsch]}
\date{\today}
\maketitle

\section{Clipping-Algorithmen}
\subsection{n-dimensionaler Liang-Barsky-Algorithmus}
\begin{itemize}
\item Im n-dimensionalen Raum werden Halbräume durch (n-1)-dimensionale Objekte definiert
\item n-
\item dd
\end{itemize}


\subsection{Sutherland-Hodgman-Algorithmus}
\begin{algorithm}[H]
 \KwData{$[P_1:P_n]$, Viewport v}
 \KwResult{In Viewport geclipptes Polygon aus der Punktliste $[P_1:P_n]$ }
 \caption{Sutherland-Hodgman-Algorithmus}
 Lise pErg\;
 \For{Jeden Eckpunkt E des Viewports v: index i}{
 	\For{Jeden Polygon-Eckpnkt P: index j}{
 		
 	   \uIf{$\overrightarrow{P_j P_{(j+1)mod n}}$ auf der sichtbaren Seite von $\overrightarrow{E_i E_{(i+1)mod4}}$}{
			$P_{(j+1)mod n}$ zu pErg hinzufügen\;
		}
		%\uElseIf{$\overrightarrow{P_j}}$ auf der sichtbaren Seite von $\overrightarrow{E_i E_{(i+1)mod4}}$ und $\overrightarrow{P_{(j+1)mod n}$ nicht}{
		%	go back to the beginning of current section\;
		%}
		\Else{}
	} 
 }
\end{algorithm}

\end{document}
