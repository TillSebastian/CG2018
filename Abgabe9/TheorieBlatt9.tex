% header
\documentclass[10pt,a4paper]{article}

\usepackage[utf8]{inputenc}
\usepackage{hyperref}
\usepackage{amssymb}
\usepackage{xcolor}
\usepackage{ngerman}
\usepackage{mathtools}
\selectlanguage{german} 
\usepackage[german,onelanguage]{algorithm2e} %for psuedo code
% the document
\begin{document}

% Ersetzt in den eckigen Klammern bitte die Übungsnummer.
\title{Abgabe - Übungsblatt [$9$]\\
\small{Einführung in die Computergraphik und Visualisierung}}
\author{ [Till Sebastian] \and [Felix Grefe] \and [Marius Rometsch]}
\date{\today}
\maketitle



\section*{Teilaufgabe 1}
\subsection*{a)}
$p'(t)=(3t^2* $ e\textsuperscript{$t^2$}$+t^3 *2t*$e\textsuperscript{$t^2$}$ ,\frac{2t*\cos (t)+t^2*\sin (t)}{\cos ^2 (t)})^T
\\=3t^2* $ e\textsuperscript{$t^2$}$+2t^4 *$e\textsuperscript{$t^2$}$ ,\frac{2t*\cos (t)+t^2*\sin (t)}{\cos ^2 (t)})^T
$
\\für t=0 gilt nun p'(t)=0 da immer mit 0 multipliziert wird.
Die Kurve ist also nicht regulär.
\subsection*{b)}

$p'(t)=(3t^2,2t+5)^T
\\T(2)=p'(2)=(3*4,4+5)^T=(12,9)^T
$
\subsection*{c)}
Eine Kurve ist Bogenlänge parametrisiert wenn die Tangente die Länge 1 hat.
\end{document}
